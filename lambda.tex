\section{Лямбда-исчисление с зависимыми типами}

\subsection{Нетипизированное лямбда-исчисление}

Лямбда-исчисление Чёрча, наряду с машинами Тьюринга и общерекурсивными функциями
Гёделя, задает множество эффективно вычислимых функций\cite{lambda-calculus-syntax-and-semantics}.
То есть функций, для вычисления которых существует алгоритм. Это, в свою очередь,
означает, что любая программа, исполняемая на современных вычислительных устройствах,
должна быть выразима на языке лямбда-исчисления. Формально этот язык задается следующим образом
\cite{lambda-calculus-syntax-and-semantics}:
\begin{itemize}
\item Есть некоторое множество \textit{переменных} \(V\).
\item Есть множество \textit{термов} \(T\), задаваемых индуктивно:
   \[
   T = \left\{
   \begin{array}{lrr}
   x,\quad& x \in V\quad&\text{переменная}\\
   M\ N,\quad& M, N \in T\quad&\text{применение}\\
   \lambda x. M,\quad& x \in V, M \in T\quad&\text{лямбда-абстракция}
   \end{array}
   \right.
   \]
\item Множеством свободных переменных терма \(t\) называется подмножество \(V\),
   в которое входят все переменные в терме \(t\), кроме тех, что связаны
   лямбда-абстракцией (определение аналогично для других связывающих
   конструкций в математике, к примеру, кванторов существования и всеобщности):
   \begin{align*}
   FV(x) &= \{x\}\\
   FV(M N) &= FV(M) \cup FV(N)\\
   FV(\lambda x. M) &= FV(M) \setminus \{x\}
   \end{align*}
\item Вычисление выполняется с помощью правила бэта-редукции:
   \[
   (\lambda x. M) N \Rightarrow_\beta M[x := N]
   \]
   Где нотация \(M[x := N]\) означает замену в \(M\) всех свободных вхождений переменной
   \(x\) на терм \(N\).
\end{itemize}

\subsection{Просто типизированное лямбда-исчисление}

\subsection{Другие способы типизирования. Куб Барендрегта}

\subsection{Зависимые типы}
