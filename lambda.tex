\section{Лямбда-исчисление с зависимыми типами}

\subsection{Нетипизированное лямбда-исчисление}

Лямбда-исчисление Чёрча, наряду с машинами Тьюринга и общерекурсивными функциями
Гёделя, задает множество эффективно вычислимых функций\cite{lambda-calculus-syntax-and-semantics}.
То есть функций, для вычисления которых существует алгоритм. Это, в свою очередь,
означает, что любая программа, исполняемая на современных вычислительных устройствах,
должна быть выразима на языке лямбда-исчисления. Формально этот язык задается следующим образом
\cite{lambda-calculus-syntax-and-semantics}:
\begin{itemize}
\item Есть некоторое множество \textit{переменных} \(V\).
\item Есть множество \textit{термов} \(T\), задаваемых индуктивно:
   \[
   T = \left\{
   \begin{array}{lrr}
   x,\quad& x \in V\quad&\text{переменная}\\
   M\ N,\quad& M, N \in T\quad&\text{применение}\\
   \lambda x. M,\quad& x \in V, M \in T\quad&\text{лямбда-абстракция}
   \end{array}
   \right.
   \]
\item Множеством свободных переменных терма \(t\) называется подмножество \(V\),
   в которое входят все переменные в терме \(t\), кроме тех, что связаны
   лямбда-абстракцией (определение аналогично для других связывающих
   конструкций в математике, к примеру, кванторов существования и всеобщности):
   \begin{align*}
   FV(x) &= \{x\}\\
   FV(M N) &= FV(M) \cup FV(N)\\
   FV(\lambda x. M) &= FV(M) \setminus \{x\}
   \end{align*}
\item Вычисление выполняется с помощью правила бэта-редукции:
   \[
   (\lambda x. M) N \Rightarrow_\beta M[x := N]
   \]
   Где нотация \(M[x := N]\) означает замену в \(M\) всех свободных вхождений переменной
   \(x\) на терм \(N\).
\end{itemize}

\subsection{Просто типизированное лямбда-исчисление}

Недостаток нетипизированного лямбда-исчисления лежит во <<вседозволенности>>
операции применения \(M N\): ничто не гарантирует, что \(M\) будет функцией.
Чтобы решить эту проблему, Чёрч разработал просто типизированное лямбда-исчисление
\cite{lambda-calculi-with-types}:
\begin{itemize}
\item Множество переменных на уровне термов \(V\) и на уровне типов \(TyV\).
\item Множество типов \(Ty\):
   \[
   Ty = \left\{
   \begin{array}{lr}
   x,\quad& x \in TyV\\
   \sigma \rightarrow \tau,\quad& \sigma, \tau \in Ty
   \end{array}
   \right.
   \]
\item Множество термов \(T\):
   \[
   T = \left\{
   \begin{array}{lr}
   x,\quad& x \in V\\
   M\ N,\quad& M, N \in T\\
   \lambda x : \sigma. M,\quad& x \in V, \sigma \in Ty, M \in T
   \end{array}
   \right.
   \]
\item Правила присваивания типов:
   \begin{align*}
   &\Gamma \vdash x : \sigma,\quad (x : \sigma) \in \Gamma\\
   &\Gamma \vdash M N : \sigma,\quad \Gamma \vdash M : \tau \rightarrow \sigma,
      \Gamma \vdash N : \tau\\
   &\Gamma \vdash (\lambda x : \sigma. M) : \sigma \rightarrow \tau,\quad
      (\Gamma \cup (x : \sigma)) \vdash M : \tau
   \end{align*}
\item Вычисление выполняется также с помощью бэта-редукции, имеющему
   аналогичный вид нетипизированному случаю. Но определяется только
   для термов, которым можно присвоить тип.
\end{itemize}

\subsection{Другие способы типизирования}

\subsection{Зависимые типы}
