\documentclass[14pt, a4paper]{extarticle}

\usepackage[top=15mm, bottom=20mm,left=30mm,right=10mm]{geometry}
\usepackage{fontspec}
\usepackage{polyglossia}
%\usepackage{amsfonts}
\usepackage{amsmath}
%\usepackage{amssymb}
%\usepackage{amsthm}
\usepackage{indentfirst} % always indent the first paragraph after section
%\usepackage{stmaryrd}
\usepackage{hyperref}
\usepackage[nottoc,numbib]{tocbibind} % include bibliography in the table of contents
\usepackage{bussproofs}
\usepackage{textgreek}

\hypersetup{pdfencoding=auto}
\defaultfontfeatures{Ligatures=TeX}
\setmainfont{CMU Serif}
\setsansfont{CMU Sans Serif}
\setmonofont{CMU Typewriter Text}

\setdefaultlanguage{russian}
\setotherlanguage{english}

\renewcommand{\baselinestretch}{1.2}

\newcommand\doubleplus{\ensuremath{\mathbin{+\mkern-5mu+}}}

\begin{document}

\begin{titlepage}

\begin{center}
Российская Академия наук\\
Санкт-Петербургский академический университет ---\\
Научно-образовательный центр нанотехнологий\\[.5cm]
Санкт-Петербургская кафедра философии РАН
\end{center}

\vspace{2cm}

\begin{center}
\Large
История и методология механизмов элиминации в языках с зависимыми типами
\end{center}

\vspace{1cm}

\begin{flushright}
Реферат аспиранта СПБАУ-НОЦНТ\\
Шабалина Александра Леонидовича\\[.5cm]

Научный руководитель\\
к.ф.-м.н., доцент\\
Москвин Денис Николаевич\\[.5cm]

Руководитель аспирантской группы\\
д.ф.н., проф.\\
Мангасарян Владимир Николаевич
\end{flushright}

\vfill

\begin{center}
Санкт-Петербург\\
2015
\end{center}

\end{titlepage}


\newpage
\setcounter{page}{2}
\tableofcontents

\newpage
\section{Лямбда-исчисление с индуктивными типами}

\subsection{Лямбда-исчисление}
\subsubsection{Нетипизированное лямбда-исчисление}

Лямбда-исчисление Чёрча, наряду с машинами Тьюринга и общерекурсивными функциями
Гёделя, задает множество эффективно вычислимых функций \cite{lambda-calculus-syntax-and-semantics}.
То есть функций, для вычисления которых существует алгоритм. Это, в свою очередь,
означает, что любая программа, исполняемая на современных вычислительных устройствах,
должна быть выразима на языке лямбда-исчисления. Формально этот язык задается следующим образом
\cite{lambda-calculus-syntax-and-semantics}:
\begin{itemize}
\item Есть некоторое множество \textit{переменных} \(V\).
\item Есть множество \textit{термов} \(T\), задаваемых индуктивно:
   \[
   T = \left\{
   \begin{array}{lrr}
   x,\quad& x \in V\quad&\text{переменная}\\
   M\ N,\quad& M, N \in T\quad&\text{применение}\\
   \lambda x. M,\quad& x \in V, M \in T\quad&\text{лямбда-абстракция}
   \end{array}
   \right.
   \]
\item Множеством свободных переменных терма \(t\) называется подмножество \(V\),
   в которое входят все переменные в терме \(t\), кроме тех, что связаны
   лямбда-абстракцией (определение аналогично для других связывающих
   конструкций в математике, к примеру, кванторов существования и всеобщности):
   \begin{align*}
   FV(x) &= \{x\}\\
   FV(M N) &= FV(M) \cup FV(N)\\
   FV(\lambda x. M) &= FV(M) \setminus \{x\}
   \end{align*}
\item Вычисление выполняется с помощью правила бэта-редукции:
   \[
   (\lambda x. M) N \Rightarrow_\beta M[x := N]
   \]
   Где нотация \(M[x := N]\) означает замену в \(M\) всех свободных вхождений переменной
   \(x\) на терм \(N\).
\end{itemize}

\subsubsection{Просто типизированное лямбда-исчисление}

Недостаток нетипизированного лямбда-исчисления лежит во <<вседозволенности>>
операции применения \(M N\): ничто не гарантирует, что \(M\) будет функцией.
Чтобы решить эту проблему, Чёрч разработал просто типизированное лямбда-исчисление
\cite{lambda-calculi-with-types}:
\begin{itemize}
\item Множество переменных на уровне термов \(V\) и на уровне типов \(TyV\).
\item Множество типов \(Ty\):
   \[
   Ty = \left\{
   \begin{array}{lr}
   x,\quad& x \in TyV\\
   \sigma \rightarrow \tau,\quad& \sigma, \tau \in Ty
   \end{array}
   \right.
   \]
\item Множество термов \(T\):
   \[
   T = \left\{
   \begin{array}{lr}
   x,\quad& x \in V\\
   M\ N,\quad& M, N \in T\\
   \lambda x : \sigma. M,\quad& x \in V, \sigma \in Ty, M \in T
   \end{array}
   \right.
   \]
\item Правила присваивания типов:
   \begin{prooftree}
   \AxiomC{}
   \RightLabel{\(\quad(x : \sigma) \in \Gamma\)}
   \UnaryInfC{\(\Gamma \vdash x : \sigma\)}
   \end{prooftree}
   \begin{prooftree}
   \AxiomC{\(\Gamma \vdash M : \tau \rightarrow \sigma\)}
   \AxiomC{\(\Gamma \vdash N : \tau\)}
   \BinaryInfC{\(\Gamma \vdash M N : \sigma\)}
   \end{prooftree}
   \begin{prooftree}
   \AxiomC{\((\Gamma \cup \{x : \sigma\}) \vdash M : \tau\)}
   \UnaryInfC{\(\Gamma \vdash (\lambda x : \sigma. M) : \sigma \rightarrow \tau\)}
   \end{prooftree}
\item Вычисление выполняется также с помощью бэта-редукции, имеющему
   аналогичный вид нетипизированному случаю. Но определяется только
   для термов, которым можно присвоить тип.
\end{itemize}

\subsubsection{Другие способы типизирования}\label{system-f}

Недостаток просто типизированного лямбда-исчисления обратный: множество термов,
которые хотелось бы написать будут отвергнуты системой типов. Поэтому было придумано
множество способов расширить систему типов, чтобы с одной стороны позволить писать
больше программ, которые хочется, а с другой --- отвергать некорректные. Одной из
таких систем типов является System F Жирара \cite{tapl}:
\begin{itemize}
\item Типы
   \[
   Ty = \left\{
   \begin{array}{lr}
   x,\quad& x \in TyV\\
   \sigma \rightarrow \tau,\quad& \sigma \in Ty, \tau \in Ty\\
   \forall x. \sigma,\quad& x \in TyV, \sigma \in Ty
   \end{array}
   \right.
   \]
\item Термы
   \[
   T = \left\{
   \begin{array}{lrr}
   x,\quad& x \in V&\\
   M\ N,\quad& M, N \in T&\\
   \lambda x : \sigma. M,\quad& x \in V, \sigma \in Ty, M \in T&\\
   \Lambda x. M,\quad& X \in TyV, M \in T\quad&\text{абстракция по типу}\\
   M [\sigma],\quad& M \in T, \sigma \in Ty\quad&\text{применение типа}
   \end{array}
   \right.
   \]
\item Типизация. Все правила из просто типизированного плюс:
   \begin{prooftree}
   \AxiomC{\(\Gamma \cup \{x\} \vdash M : \sigma\)}
   \UnaryInfC{\(\Gamma \vdash \Lambda x. M: \forall X. \sigma\)}
   \end{prooftree}
   \begin{prooftree}
   \AxiomC{\(\Gamma \vdash M : \forall x. \tau\)}
   \UnaryInfC{\(\Gamma \vdash M [\sigma] : \tau[x := \sigma]\)}
   \end{prooftree}
\end{itemize}

Введение абстракции по типам позволяет строить утверждения о программах \cite{theorems-for-free}.
Например, существует ровно один терм с типом \(\forall a. \forall b. a \rightarrow b \rightarrow a\) ---
\(\Lambda a. \Lambda b. \lambda x. \lambda y. x\).


\subsection{Сопоставление с образцом}
\subsubsection{Расширение лямбда-исчисления дополнительными типами данных}

Чтобы превратить System F в практический язык программирования,
требуется ввести дополнительные типы данных, например \(Bool\)
и \(List\) \cite{tapl}:
\begin{itemize}
\item Расширяем множество типов типами \(Bool\), \(List\ T\), где \(T \in Ty\).
\item Расширяем множество термов термами \(true\), \(false\), \(ifThenElse\),
   \(nil\), \(cons\), \(isnil\), \(head\), \(tail\).
\item Дополняем правила типизации:
   \begin{prooftree}
   \AxiomC{\(\Gamma \vdash true : Bool\)}
   \end{prooftree}
   \begin{prooftree}
   \AxiomC{\(\Gamma \vdash false : Bool\)}
   \end{prooftree}
   \begin{prooftree}
   \AxiomC{\(\Gamma \vdash e : Bool\)}
   \AxiomC{\(\Gamma \vdash t : T\)}
   \AxiomC{\(\Gamma \vdash f : T\)}
   \TrinaryInfC{\(\Gamma \vdash ifThenElse\ e\ t\ f : T\)}
   \end{prooftree}

   \begin{prooftree}
   \AxiomC{\(\Gamma \vdash nil [T] : List\ T\)}
   \end{prooftree}
   \begin{prooftree}
   \AxiomC{\(\Gamma \vdash x : T\)}
   \AxiomC{\(\Gamma \vdash xs : List\ T\)}
   \BinaryInfC{\(\Gamma \vdash cons [T]\ x\ xs : List\ T\)}
   \end{prooftree}
   \begin{prooftree}
   \AxiomC{\(\Gamma \vdash xs : List\ T\)}
   \UnaryInfC{\(\Gamma \vdash isnil [T]\ xs : Bool\)}
   \end{prooftree}
   \begin{prooftree}
   \AxiomC{\(\Gamma \vdash xs : List\ T\)}
   \UnaryInfC{\(\Gamma \vdash head [T]\ xs : T\)}
   \end{prooftree}
   \begin{prooftree}
   \AxiomC{\(\Gamma \vdash xs : List\ T\)}
   \UnaryInfC{\(\Gamma \vdash tail [T]\ xs : List\ T\)}
   \end{prooftree}
\item И расширяем правила вычислений. Назовем их дельта-редукциями:
   \begin{prooftree}
   \AxiomC{\(e \Rightarrow e'\)}
   \UnaryInfC{\(ifThenElse\ e\ t\ f \Rightarrow ifThenElse\ e'\ t\ f\)}
   \end{prooftree}
   \begin{prooftree}
   \AxiomC{\(t \Rightarrow t'\)}
   \UnaryInfC{\(ifThenElse\ true\ t\ f \Rightarrow t'\)}
   \end{prooftree}
   \begin{prooftree}
   \AxiomC{\(f \Rightarrow f'\)}
   \UnaryInfC{\(ifThenElse\ false\ t\ f \Rightarrow f'\)}
   \end{prooftree}

   \begin{prooftree}
   \AxiomC{\(x \Rightarrow x'\)}
   \AxiomC{\(xs \Rightarrow xs'\)}
   \BinaryInfC{\(cons [T]\ x\ xs \Rightarrow cons [T]\ x'\ xs'\)}
   \end{prooftree}
   \begin{prooftree}
   \AxiomC{\(xs \Rightarrow xs'\)}
   \UnaryInfC{\(isnil [T]\ xs \Rightarrow isnil [T]\ xs'\)}
   \end{prooftree}
   \begin{prooftree}
   \AxiomC{\(isnil [S]\ (nil [T]) \Rightarrow true\)}
   \end{prooftree}
   \begin{prooftree}
   \AxiomC{\(isnil [S]\ (cons [T]\ x\ xs) \Rightarrow false\)}
   \end{prooftree}
   \begin{prooftree}
   \AxiomC{\(xs \Rightarrow xs'\)}
   \UnaryInfC{\(head [T]\ xs \Rightarrow head [T]\ xs'\)}
   \end{prooftree}
   \begin{prooftree}
   \AxiomC{\(head [S]\ (cons [T]\ x\ xs) \Rightarrow x\)}
   \end{prooftree}
   \begin{prooftree}
   \AxiomC{\(xs \Rightarrow xs'\)}
   \UnaryInfC{\(tail [T]\ xs \Rightarrow tail [T]\ xs'\)}
   \end{prooftree}
   \begin{prooftree}
   \AxiomC{\(tail [S]\ (cons [T]\ x\ xs) \Rightarrow xs\)}
   \end{prooftree}
\end{itemize}

Можно заметить, что введение типа в систему состоит из шагов\label{introducing-type}:
\begin{enumerate}
\item Введение типа на уровень типов: \(Bool\), \(List\ T\).
\item Введение конструкторов --- функций (или, в частном
   случае, констант), которые дают на выходе элемент
   введенного типа: \(true\), \(false\), \(nil\), \(cons\).
\item Введение предикатов, которые проверяют каким конструктором
   был построен аргумент: \(isnil\).
\item Введение деструкторов --- функций, действующих обратно
   конструкторам, то есть получая на вход элемент вводимого типа,
   они возвращают аргумент соответствующего конструктора:
   \(head\), \(tail\)
\end{enumerate}

Последние 2 этапа можно объединить в один элиминатор, как в случае с \(Bool\):
\(ifThenElse\). В общем случае эта функция принимает на вход элемент вводимого
типа, затем столько же функций, сколько конструкторов и каждая из этих функций
имеет число аргументов такое же, как и соответствующий ей конструктор. Все эти
функции возвращают какой-то тип \(a\) и весь элиминатор тоже возвращает этот тип \(a\).
Вычислительно первый аргумент проверяется на то, каким конструктором и с какими
аргументами он был построен и на результат выдается применение соответствующей
функции к аргументам конструктора.

\subsubsection{Механизм сопоставления с образцом}

В 1968-м году Бурсталл \cite{proving-properties-of-programs-by-structural-induction}
предложил синтаксическую конструкцию, которая позволяет избавиться от явного написания
предикатов с декструкторами, заменив их соответствующими конструкторами:
\begin{align*}
&\mathbf{let}\ lst = cons\ x\ xs &\Leftrightarrow &\quad\mathbf{let}\ lst = cons\ x\ xs\\
&\mathbf{let}\ (cons\ x\ xs) = lst &\Leftrightarrow
   &\quad\mathbf{let}\ x = head\ lst; \mathbf{let}\ xs = tail\ lst\\
&lst\ \mathbf{is}\ nil &\Leftrightarrow &\quad isnil\ lst
\end{align*}
И развивая идею чуть дальше, заменять конструкции вида:
\begin{align*}
&\mathbf{if}\ &&e\ \mathbf{is}\ c_1\ \mathbf{then\ let}\ (c_1\ x_1\ \dots\ x_{k_1}) = e;
   \phi_1\ x_1\ \dots\ x_{k_1}\\
&\mathbf{else\ if}\ &&e\ \mathbf{is}\ c_2\ \mathbf{then\ let}\ (c_2\ x_1\ \dots\ x_{k_2}) = e;
   \phi_2\ x_1\ \dots\ x_{k_2}\\
&&&\vdots\\
&\mathbf{else\ if}\ &&e\ \mathbf{is}\ c_n\ \mathbf{then\ let}\ (c_n\ x_1\ \dots\ x_{k_n}) = e;
   \phi_n\ x_1\ \dots\ x_{k_n}
\end{align*}

конструкцией

\begin{align*}
&\mathbf{cases}\ e:\\
&\quad c_1\ x_1\ \dots\ x_{k_1}: \phi_1\ x_1\ \dots\ x_{k_1}\\
&\quad c_2\ x_1\ \dots\ x_{k_2}: \phi_2\ x_1\ \dots\ x_{k_2}\\
&\quad\vdots\\
&\quad c_n\ x_1\ \dots\ x_{k_n}: \phi_n\ x_1\ \dots\ x_{k_n}
\end{align*}

Механизм, названный позднее сопоставлением с образцом получил реализацию в
функциональных языках ML и Haskell. Там он оказался практически
удобным и, обретя популярность, начал появляться в новых языках программирования,
например, Rust и Swift.


\subsection{Индуктивные типы данных}
\label{inductive-definitions}
Встроенных в систему типов бывает недостаточно и хочется иметь
возможность вводить собственные типы данных. Это можно делать, явно
предоставляя функции преобразования между новым введенным типом и
комбинацией встроенных
\cite{transformation-system-for-developing-recursive-programs},
что позволяет писать программы в терминах нового типа и при этом дает возможность
изменить его внутреннее представление, не переписывая сами программы.

Но от ручного написания функций преобразования тоже хочется избавиться,
что было впервые сделано в 1980-м году в языке Hope
\cite{hope-an-experimental-applicative-language}.
В нем был введен способ вводить пользовательские типы данных, которые имеют форму деревьев.
Например, \(Bool\), \(List\ T\) принимают вид:
\begin{align*}
&\mathbf{data}\ Bool == true \doubleplus false\\
&\mathbf{typevar}\ T\\
&\mathbf{data}\ List(T) == nil \doubleplus cons(T \# List(T))
\end{align*}


\newpage
\section{Зависимые типы в лямбда-исчислении}

\subsection{Зависимые типы}
Развитие идеи из \ref{system-f} о построении утверждений по программам носит
название <<утверждения как типы>> \cite{propositions-as-types}. Это соответствие
систем типов системам математической логики. В частности, System F
соответствует интуиционистской (конструктивной) пропозициональной логике второго
порядка, но без квантора существования. Суть соответствия:

\begin{tabular}{rcl}
  Тип & \(\leftrightarrow\) & Утверждение\\
  Терм типа & \(\leftrightarrow\) & Доказательство утверждения\\
  Вычисление терма & \(\leftrightarrow\) & Упрощение доказательства
\end{tabular}

Если теперь в качестве логики взять интуиционисткую логику предикатов высших
порядков и построить по ней систему типов, то получится интуиционистская
теория типов \cite{intuitionistic-theory-of-types}, еще известная как система
зависимых типов  или просто MLTT (Martin-L\"of Type Theory, по имени автора).
<<Зависимостью>> здесь называется зависимость типов от термов:
в System F мир типов и мир термов были разделены, а в MLTT они объединены.
Примером является тип зависимых функций, \textPi-тип: \(\Pi(x : A)B(x)\).
Здесь объявляется функция с доменом \(A\) и множеством значений, зависящим от
переданного аргумента. Также в системе есть тип равенств: \(I(A, a, b)\), где \(a\) и
\(b\) имеют тип \(A\). Этот тип населен тогда и только тогда, когда \(a\) и \(b\)
равны, то есть имеют одну и ту же нормальную форму. Имея \(\Pi\) и \(I\) можно
представить, как писать верифицированные программы используя лишь систему типов:
\[
  idempotent\_sort : \Pi(x : List\ A) I(List\ A, sort\ x, sort\ (sort\ x))
\]
Функция с таким типом отвечает утверждению, что для любого списка функция \(sort\)
идемпотентна. И написав реализацию \(idempotent\_sort\), мы получаем доказательство
этого утверждение, которое будет автоматически проверено на корректность на этапе
компиляции.


\subsection{Индуктивные семейства типов}
Интенциональная теория типов Мартин-Лёфа --- закрытая система в том смысле,
что у пользователя нет способа добавить еще один тип и необходимо использовать
существующие: \textPi-типы, \textSigma-типы, размеченные объединения, тип равенства,
тип конечных множеств, натуральные числа и вселенные.

Несмотря на то, что система типов Мартин-Лёфа закрыта, введение типов данных
производилось систематически:
\begin{enumerate}
\item Ввести константу
\item Ввести правила построения термов этого типа
\item Ввести правила использования термов этого типа (элиминации)
\item Ввести правила вычисления, по которым правила элиминации и построения
  обратны друг другу
\end{enumerate}

В \cite{inductive-families} предлагает схему с семействами типов, рекурсивно
зависящами друг от друга.

Альтернативный подход: ввести тип \(W\), который является самым общим индуктивным
типом. Но мы теряем некоторые свойства связанные с равенством.


\subsection{Сопоставление с образцом для индуктивных семейств типов}
Усложнение системы по сравнению с System F сказывается на реализации сопоставления
с образцом. Наличие \textPi-типов означает, что в функции нескольких аргументов
выполняя сопоставление по первому аргументу, меняется тип последующих и, как
следствие, набор подходящих для них паттернов. А добавляя к этому еще семейства
типов, теряется линейность.

Механизм сопоставления с образцом предпочтительней прямого использования
элиминаторов по следующим соображениям \cite{dependent-pattern-matching}:
\begin{itemize}
\item Неожиданное вычислительное поведение
\item Читаемость
\item В индуктивных семействах в зависимости от того, что считать параметрами,
  а что индексами меняется элиминатор.
\end{itemize}

Первое представление сопоставления с образцом для языков с зависимыми типами и
индуктивными семействами было дано в \cite{dependent-pattern-matching}.

% TODO: What exactly does it do.


\newpage
\section{Механизмы элиминации в языках с зависимыми типами}

\subsection{Выразимость сопоставления с образцом через элиминаторы}
В \ref{dependent-pattern-matching} сопоставление с образцом приводилось как
альтернатива элиминаторам. С точки зрения использования оно удобнее, потому
что напоминает математическое задание функции и из определения сразу видно
вычислительную составляющую \cite{dependent-pattern-matching}.
Но с точки зрения теории, механизм усложняет доказательство корректности ядра
системы.

Чтобы разрешать сопоставление с образцом, но при этом оставить ядро простым для
понимания в \cite{eliminating-dependent-pattern-matching} предлагается способ
трансляции функции, заданной с помощью сопоставления с образцом в функцию,
заданную с помощью элиминаторов, сохраняя вычислительные свойства.

Еще один результат этой работы --- доказательство, что для введения в MLTT
сопоставления с образцом необходимо и достаточно обогатить систему аксиомой
\(K\). Аксиома \(K\) --- это альтернативный элиминатор для типа равенства.
Он имеет тип:
\[
K : \Pi(x : A) \Pi(c : C(r(x))) \Pi(h : Id(A, x, x)) C(h)
\]
Элиминатор, введенный в \ref{inductive-families} (называемый \(J\)) имеет тип:
\[
J : \Pi(x : A) \Pi(c : C(x, x, r(x))) \Pi(y : A) \Pi(z : I(A, x, y)) C(x, y, z)
\]
Из них двоих можно получить:
\begin{align*}
&UIP : \Pi(x : A) \Pi(y : A) \Pi(p_1 : Id(A, x, y)) \Pi(p_2 : Id(A, x, y)) Id(Id(A, x, y), p_1, p_2)\\
&UIP(x, y, p_1, p_2) = J(x, \lambda q.\ K(x, r(r(x)), q), y, r(r(x)))(p_2)
\end{align*}
Другими словами, что все доказательства равенства \(x\) и \(y\) равны между собой.
Такое свойство совместимо с MLTT, но вступает в противоречие с набирающей популярность
гомотопической теорией типов (HoTT) \cite{homotopy-type-theory}. HoTT можно
рассматривать как расширение MLTT добавлением аксиомы унивалентности и высших
индуктивных типов. Отсюда следует, что в HoTT не может быть использован метод
сопоставления с образцом из \cite{dependent-pattern-matching}. Тем не менее,
чтобы не терять механизм полностью, предлагаются варианты ограничения механизма,
так чтобы он не требовал \(K\) \cite{pattern-matching-without-K}.


\subsection{Вариации сопоставления с образцом}
\subsubsection{Механизм <<smartcase>>}

Решение задачи унификации \(x\) и \(y\) при сопоставлении с образцом в
\cite{dependent-pattern-matching} редуцирует оба до нормальной формы и затем
проверяет соответствие конструкторов друг другу. Но поскольку при унификации
решается задача проверки на равенство, то можно было бы использовать другие
свойства равенства: симметричность, транзитивность и конгруэнтность, беря уже
существующие равенства из контекста. Но это делает задачу неразрешимой
\cite{programming-up-to-congruence}. В \cite{programming-up-to-congruence}
исследуется система, в которой отказались от автоматической бэта-редукции в
пользу конгруэнтного замыкания.

В частности, при сопоставлении с образцом, сопоставление на каком-то аргументе
добавляет новые равенства в контекст, которые могут быть использованы в будущем.
При обычной реализации в \cite{dependent-pattern-matching} порожденные уравнения
будут использованы для переписывания контекста и тут же забыты.

\subsubsection{Сопоставление с образцом без упорядочивания уравнений}

Механизмы в \ref{pattern-matching} и \ref{dependent-pattern-matching} используют
упорядоченный набор уравнений: алгоритм идет сверху вниз по уравнениям, пока не
найдет первый подходящий набор образцов. Преимущества такого способа в том,
что он позволяет в некоторых случаях уменьшить количество требуемых уравнений
\cite{overlapping-and-order-independent-patterns} и в том, что написанные таким
образом функции подлежат трансляции в элиминаторы как описано в
\cite{eliminating-dependent-pattern-matching}.

С другой стороны, при таком определении отсутствует интересное свойство:
уравнения нельзя считать равенствами по определению. Уравнение для набора
образцов является равенством в системе только если все предыдущие наборы
образцов не подошли. Например, определение сложения натуральных чисел:
\begin{align*}
&add : Nat \to Nat \to Nat\\
&add\ 0\ n = n\\
&add\ (succ\ n)\ m = succ\ (add\ n\ m)\\
&add\ n\ 0 = n\\
&add\ n\ (succ\ m) = succ\ (add\ n\ m)
\end{align*}
Из этого определения не следует, что \(add\ n\ 0 =_{def} n\). Этот факт придется
доказывать отдельно и система не сможет использовать его автоматически.

В \cite{overlapping-and-order-independent-patterns} предлагается использовать
подход, когда каждое уравнение порождает равенство по определению.


\subsection{Проблемы механизма сопоставлением с образцом}
С точки зрения удобства программирования механизм сопоставления с образцом
несовершенен. Описанный в \ref{dependent-pattern-matching} метод разбиения
контекста и расширенный в \ref{the-view-from-the-left}, переписывает контекст
незаметно для пользователя. Точнее, чтения кода недостаточно, требуется помощь
утилит, чтобы анализировать, каким будет контекст в данной точке кода.

Зависимость левых аргументов от правых, вызванная индексами в типе правых приводит
к ситуации, когда при написании кода приходится возвращаться назад и исправлять
уже написанный.

А поскольку равенство, используемое при унификации интенционально, то некоторые
ожидаемые равенства (например, коммутативность сложения) не могут быть применены
автоматически и приходится изменять тип функции, вытаскивая равенство в дополнительный
аргумент, прося пользователя подставить необходимое доказательство.

А так как унификация неразрешима, то некоторые правильные определения функций
будут недопущены.


\newpage
\bibliographystyle{plain}
\bibliography{bibliography}

\end{document}
