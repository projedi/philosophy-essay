\section{Индуктивные типы данных}

\subsection{Расширение лямбда-исчисления дополнительными типами данных}

Чтобы превратить лямбда-исчисление в практический язык программирования,
требуется ввести дополнительные типы данных, такие как числа, списки, \dots.
Здесь будем рассматривать System F.

Посмотрим, как можно ввести типы \(Bool\), \(Nat\) и \(List\)\cite{tapl}:
\begin{itemize}
\item Расширяем множество типов типами \(Bool\), \(Nat\), \(List\ T\), где \(T \in Ty\).
\item Расширяем множество термов термами \(true\), \(false\), \(ifThenElse\),
   \(0\), \(succ\), \(pred\), \(iszero\),
   \(nil\), \(cons\), \(isnil\), \(head\), \(tail\).
\item Дополняем правила типизации:
   \begin{prooftree}
   \AxiomC{\(\Gamma \vdash true : Bool\)}
   \end{prooftree}
   \begin{prooftree}
   \AxiomC{\(\Gamma \vdash false : Bool\)}
   \end{prooftree}
   \begin{prooftree}
   \AxiomC{\(\Gamma \vdash e : Bool\)}
   \AxiomC{\(\Gamma \vdash t : T\)}
   \AxiomC{\(\Gamma \vdash f : T\)}
   \TrinaryInfC{\(\Gamma \vdash ifThenElse\ e\ t\ f : T\)}
   \end{prooftree}

   \begin{prooftree}
   \AxiomC{\(\Gamma \vdash 0 : Nat\)}
   \end{prooftree}
   \begin{prooftree}
   \AxiomC{\(\Gamma \vdash n : Nat\)}
   \UnaryInfC{\(\Gamma \vdash succ\ n : Nat\)}
   \end{prooftree}
   \begin{prooftree}
   \AxiomC{\(\Gamma \vdash n : Nat\)}
   \UnaryInfC{\(\Gamma \vdash pred\ n : Nat\)}
   \end{prooftree}
   \begin{prooftree}
   \AxiomC{\(\Gamma \vdash n : Nat\)}
   \UnaryInfC{\(\Gamma \vdash iszero\ n : Bool\)}
   \end{prooftree}

   \begin{prooftree}
   \AxiomC{\(\Gamma \vdash nil [T] : List\ T\)}
   \end{prooftree}
   \begin{prooftree}
   \AxiomC{\(\Gamma \vdash x : T\)}
   \AxiomC{\(\Gamma \vdash xs : List\ T\)}
   \BinaryInfC{\(\Gamma \vdash cons [T]\ x\ xs : List\ T\)}
   \end{prooftree}
   \begin{prooftree}
   \AxiomC{\(\Gamma \vdash xs : List\ T\)}
   \UnaryInfC{\(\Gamma \vdash isnil [T]\ xs : Bool\)}
   \end{prooftree}
   \begin{prooftree}
   \AxiomC{\(\Gamma \vdash xs : List\ T\)}
   \UnaryInfC{\(\Gamma \vdash head [T]\ xs : T\)}
   \end{prooftree}
   \begin{prooftree}
   \AxiomC{\(\Gamma \vdash xs : List\ T\)}
   \UnaryInfC{\(\Gamma \vdash tail [T]\ xs : List\ T\)}
   \end{prooftree}
\item И расширяем правила вычислений. Назовем их дельта-редукциями:
   \begin{prooftree}
   \AxiomC{\(e \Rightarrow e'\)}
   \UnaryInfC{\(ifThenElse\ e\ t\ f \Rightarrow ifThenElse\ e'\ t\ f\)}
   \end{prooftree}
   \begin{prooftree}
   \AxiomC{\(t \Rightarrow t'\)}
   \UnaryInfC{\(ifThenElse\ true\ t\ f \Rightarrow t'\)}
   \end{prooftree}
   \begin{prooftree}
   \AxiomC{\(f \Rightarrow f'\)}
   \UnaryInfC{\(ifThenElse\ false\ t\ f \Rightarrow f'\)}
   \end{prooftree}

   \begin{prooftree}
   \AxiomC{\(n \Rightarrow n'\)}
   \UnaryInfC{\(succ\ n \Rightarrow succ\ n'\)}
   \end{prooftree}
   \begin{prooftree}
   \AxiomC{\(n \Rightarrow n'\)}
   \UnaryInfC{\(pred\ n \Rightarrow pred\ n'\)}
   \end{prooftree}
   \begin{prooftree}
   \AxiomC{\(pred\ (succ\ n) \Rightarrow n\)}
   \end{prooftree}
   \begin{prooftree}
   \AxiomC{\(n \Rightarrow n'\)}
   \UnaryInfC{\(iszero\ n \Rightarrow iszero\ n'\)}
   \end{prooftree}
   \begin{prooftree}
   \AxiomC{\(iszero\ 0 \Rightarrow true\)}
   \end{prooftree}
   \begin{prooftree}
   \AxiomC{\(iszero\ (succ\ n) \Rightarrow false\)}
   \end{prooftree}

   \begin{prooftree}
   \AxiomC{\(x \Rightarrow x'\)}
   \AxiomC{\(xs \Rightarrow xs'\)}
   \BinaryInfC{\(cons [T]\ x\ xs \Rightarrow cons [T]\ x'\ xs'\)}
   \end{prooftree}
   \begin{prooftree}
   \AxiomC{\(xs \Rightarrow xs'\)}
   \UnaryInfC{\(isnil [T]\ xs \Rightarrow isnil [T]\ xs'\)}
   \end{prooftree}
   \begin{prooftree}
   \AxiomC{\(isnil [S]\ (nil [T]) \Rightarrow true\)}
   \end{prooftree}
   \begin{prooftree}
   \AxiomC{\(isnil [S]\ (cons [T]\ x\ xs) \Rightarrow false\)}
   \end{prooftree}
   \begin{prooftree}
   \AxiomC{\(xs \Rightarrow xs'\)}
   \UnaryInfC{\(head [T]\ xs \Rightarrow head [T]\ xs'\)}
   \end{prooftree}
   \begin{prooftree}
   \AxiomC{\(head [S]\ (cons [T]\ x\ xs) \Rightarrow x\)}
   \end{prooftree}
   \begin{prooftree}
   \AxiomC{\(xs \Rightarrow xs'\)}
   \UnaryInfC{\(tail [T]\ xs \Rightarrow tail [T]\ xs'\)}
   \end{prooftree}
   \begin{prooftree}
   \AxiomC{\(tail [S]\ (cons [T]\ x\ xs) \Rightarrow xs\)}
   \end{prooftree}
\end{itemize}

Можно заметить, что введение типа в систему состоит из шагов:
\begin{enumerate}
\item Введение типа на уровень типов: \(Bool\), \(Nat\), \(List\ T\).
\item Введение конструкторов --- функций (или, в частном
   случае, констант), которые дают на выходе элемент
   введенного типа: \(true\), \(false\), \(0\), \(succ\), \(nil\), \(cons\).
\item Введение предикатов, которые проверяют каким конструктором
   был построен аргумент: \(iszero\), \(isnil\).
\item Введение деструкторов --- функций, действующих обратно
   конструкторам, то есть получая на вход элемент вводимого типа,
   они возвращают аргумент соответствующего конструктора:
   \(pred\), \(head\), \(tail\)
\end{enumerate}

Последние 2 этапа можно объединить в один элиминатор, как в случае с \(Bool\):
\(ifThenElse\). В общем случае эта функция принимает на вход элемент вводимого
типа, затем столько же функций, сколько конструкторов и каждая из этих функций
имеет число аргументов такое же, как и соответствующий ей конструктор. Все эти
функции возвращают какой-то тип \(a\) и весь элиминатор тоже возвращает этот тип \(a\).
Вычислительно первый аргумент проверяется на то, каким конструктором и с какими
аргументами он был построен и на результат выдается применение соответствующей
функции к аргументам конструктора.

\subsection{Механизм сопоставления с образцом}

TODO: Из \cite{proving-properties-of-programs-by-structural-induction}.

\subsection{Алгебраические типы данных}

TODO: Из \cite{transformation-system-for-developing-recursive-programs}.

\subsection{Зависимые индуктивные типы данных}

TODO: Calculus of Inductive Constructions

TODO: UTT
