В \ref{dependent-pattern-matching} сопоставление с образцом приводилось как
альтернатива элиминаторам. С точки зрения использования оно удобнее, потому
что напоминает математическое задание функции и из определения сразу видно
вычислительную составляющую \cite{dependent-pattern-matching}.
Но с точки зрения теории, механизм усложняет доказательство корректности ядра
системы.

Чтобы разрешать сопоставление с образцом, но при этом оставить ядро простым для
понимания в \cite{eliminating-dependent-pattern-matching} предлагается способ
трансляции функции, заданной с помощью сопоставления с образцом в функцию,
заданную с помощью элиминаторов, сохраняя вычислительные свойства.

Еще один результат этой работы --- доказательство, что для введения в MLTT
сопоставления с образцом необходимо и достаточно обогатить систему аксиомой
\(K\). Аксиома \(K\) --- это альтернативный элиминатор для типа равенства.
Он имеет тип: \(K : P(r(x)) \to \Pi(h : Id(A, x, x)) P(h)\). Отсюда в том
числе следует \(UIP : \Pi(p_1 : Id(A, x, y)) \Pi(p_2 : Id(A, x, y)) Id(Id(A, x, y), p_1, p_2)\).
Другими словами, что все доказательства равенства \(x\) и \(y\) равны между собой.
Такое свойство совместимо с MLTT, но вступает в противоречие с набирающей популярность
гомотопической теорией типов (HoTT) \cite{homotopy-type-theory}. HoTT можно
рассматривать как расширение MLTT добавлением аксиомы унивалентности и высших
индуктивных типов. Отсюда следует, что в HoTT не может быть использован метод
сопоставления с образцом из \cite{dependent-pattern-matching}. Тем не менее,
чтобы не терять механизм полностью, предлагаются варианты ограничения механизма,
так чтобы он не требовал \(K\) \cite{pattern-matching-without-K}.
