С точки зрения удобства программирования механизм сопоставления с образцом
несовершенен. Описанный в \ref{dependent-pattern-matching} метод разбиения
контекста и расширенный в \ref{the-view-from-the-left}, переписывает контекст
незаметно для пользователя. Точнее, чтения кода недостаточно, требуется помощь
утилит, чтобы анализировать, каким будет контекст в данной точке кода.

Зависимость левых аргументов от правых, вызванная индексами в типе правых приводит
к ситуации, когда при написании кода приходится возвращаться назад и исправлять
уже написанный.

А поскольку равенство, используемое при унификации интенционально, то некоторые
ожидаемые равенства (например, коммутативность сложения) не могут быть применены
автоматически и приходится изменять тип функции, вытаскивая равенство в дополнительный
аргумент, прося пользователя подставить необходимое доказательство.

А так как унификация неразрешима, то некоторые правильные определения функций
будут недопущены.
