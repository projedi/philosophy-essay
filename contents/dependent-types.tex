Развитие идеи из \ref{system-f} о построении утверждений по программам носит
название <<утверждения как типы>> \cite{propositions-as-types}. Это соответствие
систем типов системам математической логики. В частности, System F
соответствует интуиционистской (конструктивной) пропозициональной логике второго
порядка, но без квантора существования. Суть соответствия:

\begin{tabular}{rcl}
  Тип & \(\leftrightarrow\) & Утверждение\\
  Терм типа & \(\leftrightarrow\) & Доказательство утверждения\\
  Вычисление терма & \(\leftrightarrow\) & Упрощение доказательства
\end{tabular}

Если теперь в качестве логики взять интуиционисткую логику предикатов высших
порядков и построить по ней систему типов, то получится интуиционистская
теория типов \cite{intuitionistic-theory-of-types}, еще известная как система
зависимых типов  или просто MLTT (Martin-L\"of Type Theory, по имени автора).
<<Зависимостью>> здесь называется зависимость типов от термов:
в System F мир типов и мир термов были разделены, а в MLTT они объединены.
Примером является тип зависимых функций, \textPi-тип: \(\Pi(x : A)B(x)\).
Здесь объявляется функция с доменом \(A\) и множеством значений, зависящим от
переданного аргумента. Также в системе есть тип равенств: \(I(A, a, b)\), где \(a\) и
\(b\) имеют тип \(A\). Этот тип населен тогда и только тогда, когда \(a\) и \(b\)
равны, то есть имеют одну и ту же нормальную форму. Имея \(\Pi\) и \(I\) можно
представить, как писать верифицированные программы используя лишь систему типов:
\[
  idempotent\_sort : \Pi(x : List\ A) I(List\ A, sort\ x, sort\ (sort\ x))
\]
Функция с таким типом отвечает утверждению, что для любого списка функция \(sort\)
идемпотентна. И написав реализацию \(idempotent\_sort\), мы получаем доказательство
этого утверждение, которое будет автоматически проверено на корректность на этапе
компиляции.

Нужно заметить, что поскольку в типах теперь могут встречаться термы, то требуется
гарантия, что проверка типов разрешима. Для этого достаточно, чтобы все термы
редуцировались до нормальной формы за конечное число шагов.
