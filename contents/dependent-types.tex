Развивая идею построения утверждения по программам, можно придти к соответствию Карри-Говарда
между типами и теоремами в математической логике \cite{propositions-as-types}. В частности, System F
соответствует интуиционистской (конструктивной) пропозициональной логике второго порядка
(но без квантора существования).

Если теперь вместо пропозициональной логики взять логику предикатов и построить по ней систему типов,
то получатся так называемые зависимые типы \cite{intuitionistic-theory-of-types}.
Ключевая идея в том, что мы объединяем множества типов и термов в одно. Позволяя таким образом писать
термы на уровне типов. Это открывает возможность для проведения формальной верификации программ с
помощью одной лишь системы типов.
