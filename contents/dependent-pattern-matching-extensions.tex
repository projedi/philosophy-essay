\label{dependent-pattern-matching-extensions}

\subsubsection{Механизм <<views>>}\label{the-view-from-the-left}

В \cite{views} Уодлер представляет способ создавать различные виды для
типов данных, чтобы затем выполнять сопоставление с образцом по виду. Таким
образом, выполняя весь анализ в левой части уравнения, а в правой оставляя только
результат. В частности:
\begin{align*}
&view\ int ::= Zero\ |\ Even\ int\ |\ Odd\ int\\
&\quad in\ n = Zero,\quad if\ n = 0\\
&\qquad = Even\ (n\ div\ 2),\quad if\ n > 0 \land n\ mod\ 2 = 0\\
&\qquad = Odd\ ((n - 1)\ div\ 2),\quad if\ n > 0 \land n\ mod\ 2 = 1\\
&\quad out\ Zero = 0\\
&\quad out\ (Even\ n) = 2 * n,\quad if\ 2 * n > 0\\
&\quad out\ (Odd\ n) = 2 * n + 1,\quad if\ 2 * n + 1 > 0\\
\\
&power\ x\ Zero = 1\\
&power\ x\ (Even\ n) = power\ (x * x)\ n\\
&power\ x\ (Odd\ n) = x * power\ (x * x)\ n
\end{align*}

В языке Haskell используется похожий механизм, но дополнительная структура с
описанием вида не создается --- вместо нее функции:
\begin{align*}
&even\ n\\
&|\ n > 0\\
&,\ n\ `mod`\ 2 == 0 = Just\ (n\ `div`\ 2)\\
\\
&odd\ n\\
&|\ n > 0\\
&,\ n\ `mod`\ 2 == 1 = Just\ ((n - 1)\ `div`\ 2)\\
\\
&power\ x\ 0 = 1\\
&power\ x\ n\\
&|\ Just\ k \leftarrow even\ n = power\ (x * x)\ k\\
&|\ Just\ k \leftarrow odd\ n = x * power\ (x * x)\ k
\end{align*}

Расширение этого механизма для зависимых типов рассматривается в
\cite{the-view-from-the-left}. По аналогии с тем, как
изменяется сопоставление с образцом при наличии зависимых типов,
изменяется и механизм видов. А именно, он уточняет контекст.

В частности:
\begin{align*}
&data\ Parity\ (n : Nat) : Set\ where\\
&\quad even : (k : Nat) \to Parity\ (2 * k)\\
&\quad odd : (k : Nat) \to Parity\ (1 + 2 * k)\\
\\
&parity : (n : Nat) \to Parity n\\
&parity\ 0 = even\ 0\\
&parity\ 1 = odd\ 0\\
&parity\ (succ\ (succ\ n))\ with\ parity\ n\\
&parity\ .\_\ |\ even\ k = even\ (succ\ k)\\
&parity\ .\_\ |\ odd\ k = odd\ (succ\ k)\\
\\
&power\ x\ 0 = 1\\
&power\ x\ n\ with\ parity\ n\\
&power\ x\ .(2 * k)\ |\ even\ k = power\ (x * x)\ k\\
&power\ x\ .(1 + 2 * k)\ |\ odd\ k = x * power\ (x * x)\ k
\end{align*}

Помимо уточнения контекста наличие индуктивных семейств позволяет
задавать виды, корректные по построению. В \cite{views} отсутствие
мощной системы типов заставляет полагаться на совесть программиста:
при задании вида \(in\) и \(out\) должны быть взаимообратны.

\subsubsection{Механизм <<smartcase>>}

Решение задачи унификации \(x\) и \(y\) при сопоставлении с образцом в
\cite{dependent-pattern-matching} редуцирует оба до нормальной формы и затем
проверяет соответствие конструкторов друг другу. Но поскольку при унификации
решается задача проверки на равенство, то можно было бы использовать другие
свойства равенства: симметричность, транзитивность и конгруэнтность, беря уже
существующие равенства из контекста. Но это делает задачу неразрешимой
\cite{programming-up-to-congruence}. В \cite{programming-up-to-congruence}
исследуется система, в которой отказались от автоматической бэта-редукции в
пользу конгруэнтного замыкания.

В частности, при сопоставлении с образцом, сопоставление на каком-то аргументе
добавляет новые равенства в контекст, которые могут быть использованы в будущем.
При обычной реализации в \cite{dependent-pattern-matching} порожденные уравнения
будут использованы для переписывания контекста и тут же забыты.

\subsubsection{Сопоставление с образцом без упорядочивания уравнений}

Механизмы в \ref{pattern-matching} и \ref{dependent-pattern-matching} используют
упорядоченный набор уравнений: алгоритм идет сверху вниз по уравнениям, пока не
найдет первый подходящий набор образцов. Преимущества такого способа в том,
что он позволяет в некоторых случаях уменьшить количество требуемых уравнений
\cite{overlapping-and-order-independent-patterns} и в том, что написанные таким
образом функции подлежат трансляции в элиминаторы как описано в
\cite{eliminating-dependent-pattern-matching}.

С другой стороны, при таком определении отсутствует интересное свойство:
уравнения нельзя считать равенствами по определению. Уравнение для набора
образцов является равенством в системе только если все предыдущие наборы
образцов не подошли. Например, определение сложения натуральных чисел:
\begin{align*}
&add : Nat \to Nat \to Nat\\
&add\ 0\ n = n\\
&add\ (succ\ n)\ m = succ\ (add\ n\ m)\\
&add\ n\ 0 = n\\
&add\ n\ (succ\ m) = succ\ (add\ n\ m)
\end{align*}
Из этого определения не следует, что \(add\ n\ 0 =_{def} n\). Этот факт придется
доказывать отдельно и система не сможет использовать его автоматически.

В \cite{overlapping-and-order-independent-patterns} предлагается использовать
подход, когда каждое уравнение порождает равенство по определению.
