Встроенных в систему типов бывает недостаточно и хочется иметь
возможность вводить собственные типы данных. Это можно делать, явно
предоставляя функции преобразования между новым введенным типом и
комбинацией встроенных
\cite{transformation-system-for-developing-recursive-programs},
что позволяет писать программы в терминах нового типа и при этом дает возможность
изменить его внутреннее представление, не переписывая сами программы.

Но от ручного написания функций преобразования тоже хочется избавиться,
что было впервые сделано в 1980-м году в языке Hope
\cite{hope-an-experimental-applicative-language}.
В нем был введен способ вводить пользовательские типы данных, которые имеют форму деревьев.
Например, \(Bool\), \(List\ T\) принимают вид:
\begin{align*}
&\mathbf{data}\ Bool == true \doubleplus false\\
&\mathbf{typevar}\ T\\
&\mathbf{data}\ List(T) == nil \doubleplus cons(T \# List(T))
\end{align*}
