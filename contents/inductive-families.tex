MLTT --- закрытая система в том смысле, что у пользователя нет способа добавить
еще один тип и необходимо использовать существующие: \textPi-типы,
\textSigma-типы, размеченные объединения, тип равенства, тип конечных множеств,
натуральные числа и вселенные. Преимущество закрытой системы в её простоте
\cite{observational-equality-now}: легче доказать непротиворечивость и прочие свойства.
Но как отмечалось в \ref{inductive-definitions}, встроенных типов бывает недостаточно.

Один из подходов --- расширить систему одним типом \(W\), через который выразимы
все индуктивные типы \cite{inductive-families}. Но он работает только в системах
с экстенциональным равенством, а рассматриваемая здесь система обладает интенциональным.
Интенциональное равенство --- равенство по определению с точностью до нормальной
формы и, вследствие редуцируемости всех термов, является разрешимым. Экстенциональное
равенство --- равенство по наблюдению. В частности, функции экстенционально равны,
когда они на всех аргументах дают одинаковый результат, а интенционально равны,
когда их определения совпадают. Недостаток эксенциональности в неразрешимости
\cite{observational-equality-now}.

Другой подход заключается в открытии наружу схемы введения новых типов.
Несмотря на закрытость системы, Мартин-Лёф вводил каждый новый тип по одной
и той же схеме, которая напоминает используемую в \ref{introducing-type}:
\begin{enumerate}
\item Ввести константу, отвечающую типу
\item Ввести правила построения термов этого типа
\item Ввести правила использования термов этого типа (элиминации)
\item Ввести правила вычисления, по которым правило элиминации обратно
  правилу построения
\end{enumerate}

Например, равенство определяется:
\begin{enumerate}
\item \(I(A, a, b)\), где \(a : A\), \(b : A\).
\item \(r : \Pi(x : A)I(A, x, x)\)
\item \(I\_elim : \Pi(x : A) \Pi(c : C(x, x, r(x))) \Pi(y : A) \Pi(z : I(A, x, y)) C(x, y, z)\)
\item \(I\_elim(x, c, x, r(x)) = c\).
\end{enumerate}

Равенство еще является примером семейств индуктивных типов: в \(I(A, a, b)\)
у типа три параметра: \(A\), \(a\) и \(b\). У конструктора \(r\) результирующий
тип \(I(A, x, x)\), где \(x\) --- аргумент \(r\). В такой ситуации \(A\) называется
параметром типа, потому что во всех конструкторах он получен из определения
типа, а \(a\) и \(b\) называются индексами, потому что они зависят от конструктора
и его аргументов. Типы с индексами называются семействами типов.

Дибиер \cite{inductive-families} расширяет эту схему семействами типов,
которые могут рекурсивно зависеть друг от друга.
