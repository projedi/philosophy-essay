Усложнение системы по сравнению с System F сказывается на реализации сопоставления
с образцом. Наличие \textPi-типов означает, что в функции нескольких аргументов
выполняя сопоставление по первому аргументу, меняется тип последующих и, как
следствие, набор подходящих для них паттернов. А добавляя к этому еще семейства
типов, теряется линейность.

Механизм сопоставления с образцом предпочтительней прямого использования
элиминаторов по следующим соображениям \cite{dependent-pattern-matching}:
\begin{itemize}
\item Неожиданное вычислительное поведение
\item Читаемость
\item В индуктивных семействах в зависимости от того, что считать параметрами,
  а что индексами меняется элиминатор.
\end{itemize}

Первое представление сопоставления с образцом для языков с зависимыми типами и
индуктивными семействами было дано в \cite{dependent-pattern-matching}.

% TODO: What exactly does it do.
