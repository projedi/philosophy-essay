\label{dependent-pattern-matching}
Усложнение системы по сравнению с System F сказывается на реализации сопоставления
с образцом. Наличие \textPi-типов означает, что в функции нескольких аргументов
выполняя сопоставление по первому аргументу, меняется тип последующих и, как
следствие, набор подходящих для них паттернов. А добавляя к этому еще семейства
типов, теряется линейность. Линейным называется паттерн, в котором все переменные
различны. Наличие типов с индексами может наложить ограничения
на две различные переменные в паттерне, требуя чтобы они были одинаковы. Например,
\[
f : \Pi(n : Nat) \Pi(k : Nat) \Pi(pr : I(Nat, n + 1, k)) Nat
\]
Выполняя
сопоставление с образцом на третьем аргументе, единственный подходящий конструктор
--- \(r(n + 1)\). Его тип \(I(Nat, n + 1, n + 1)\) и таким образом на месте второго
аргумента обязано стоять выражение \(n + 1\).

Первое представление сопоставления с образцом для языков с зависимыми типами и
индуктивными семействами было дано в \cite{dependent-pattern-matching}.

Идея состоит в разбиении контекста при сопоставлении аргумента с образцом.
Имея функцию
\[
f : \Pi (x_1 : A_1) \dots \Pi (x_n : A_n) : A
\]
контекстом является набор \(x_i : A_i\) и \(A\). Сопоставление с одним из \(x_i\)
разбивает контекст: уточняется значение \(x_i\), тип \(A_i\) (если в нем есть
индексы), все \(A_j\) для \(j > i\) и \(A\) из-за возможной зависимости от \(x_i\),
все \(x_j\) для \(j < i\) из-за возможной зависимости по индексам в \(A_i\).
Уточнение \(A_j\) приводит к тому, что множество их значений может измениться
и в том числе стать пустым. В таком случае тело функции для данного образца
можно не реализовывать, потому что все равно невозможно было бы вызвать функцию
с такими аргументами.

Поскольку в MLTT все функции обязаны завершаться на всех входах, то для
данного способа введения новых функциональных констант в систему нужно ввести
ограничения: для каждого входа должен быть соответствующий образец и никакой
рекурсивный вызов не должен привести к бесконечной рекурсии. Обе эти задачи
в общем случае неразрешимы \cite{eliminating-dependent-pattern-matching}. И
поэтому в обоих случаях используются неточные алгоритмы.
